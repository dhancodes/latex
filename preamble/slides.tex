\documentclass{beamer}
\usepackage[utf8]{inputenc}
\usepackage[english]{babel}
\usepackage{amssymb}
\usepackage{cite,natbib} %For better bibliography.
\usepackage{graphicx}
\usepackage{mathtools}
\usepackage{tikz-cd} %For commutative diagrams.
\usetikzlibrary{matrix}

\usetheme[block=fill,numbering=fraction,progressbar=frametitle]{metropolis}
\useinnertheme{circles}

%\setbeamertemplate{theorems}[numbered]

\usepackage{thmtools} %for spacing of theroem heading in title environment.
\declaretheoremstyle[
headfont=\bfseries,%
headpunct={\vspace{\topsep}\newline}, %
numbered=no,
spaceabove=3\topsep, %
postheadspace=0 pt ]{exercs}


%For better bibliography.
\bibliographystyle{unsrt}
\renewcommand{\bibfont}{\footnotesize}
\setbeamertemplate{frametitle continuation}[from second]

\newcommand{\bbox}[1]{
	\noindent\fbox{ \parbox{\textwidth}{#1} \vspace{2pt}} }
\renewcommand{\bibsection}{}

%Theorem environments
\usepackage{mdframed}
\mdfsetup{skipabove=1em,skipbelow=0em}
\theoremstyle{definition}
\newmdtheoremenv[nobreak=true]{lmma}{Lemma}
\newmdtheoremenv{conclusion}{Conclusion}
\newtheorem{defn}{Definition}
\newmdtheoremenv{prop}{Proposition}
\newmdtheoremenv{lma}{Lemma}
\newmdtheoremenv{cor}{Corollary}
\newmdtheoremenv{thm}{Theorem}
\newtheorem*{obs}{Observation}
\theoremstyle{remark}
\newtheorem*{rmk}{Remark}
\newtheorem*{eg}{Example}

% Some shortcuts
\newcommand\N{\ensuremath{\mathbb{N}}}
\newcommand\R{\ensuremath{\mathbb{R}}}
\newcommand\Z{\ensuremath{\mathbb{Z}}}
\renewcommand\O{\ensuremath{\emptyset}}
\newcommand\Q{\ensuremath{\mathbb{Q}}}
\renewcommand\C{\ensuremath{\mathbb{C}}}
\newcommand\F{\ensuremath{\mathbb{F}}}
\newcommand{\Zn}{\ensuremath{\mathbb{Z}_{\geq 0}^n}}
\newcommand{\m}{\ensuremath{\mathfrak{m}}}
\renewcommand{\S}{\ensuremath{\mathcal{S}}}
\renewcommand{\a}{\ensuremath{\mathfrak{a}}}
\renewcommand{\b}{\ensuremath{\mathfrak{b}}}

%Make implies and impliedby shorter
\let\implies\Rightarrow
\let\impliedby\Leftarrow
\let\iff\Leftrightarrow
\let\epsilon\varepsilon
\let\phi\varphi

% Add \contra symbol to denote contradiction
\usepackage{stmaryrd} % for \lightning
\newcommand\contra{\scalebox{1.5}{$\lightning$}}

\usepackage{thmtools}
\usepackage{kantlipsum}

\declaretheoremstyle[
headfont=\bfseries,%
headpunct={\vspace{\topsep}\newline}, %
numbered=no,
spaceabove=3\topsep, %
postheadspace=0 pt ]{exercs}
\declaretheorem[name=EXERCISES,style=exercs]{problems}

\usepackage{todonotes}
\usepackage[most]{tcolorbox}

\newtcolorbox{noot}[1][]{enhanced,
  before skip=2mm,after skip=3mm,
  boxrule=0.4pt,left=5mm,right=2mm,top=1mm,bottom=1mm,
  colback=white!50,
  colframe=yellow!20!black,
  sharp corners,rounded corners=southeast,arc is angular,arc=3mm,
  underlay={%
    \path[fill=black] ([yshift=3mm]interior.south east)--++(-0.4,-0.1)--++(0.1,-0.2);
    \path[draw=black,shorten <=-0.05mm,shorten >=-0.05mm] ([yshift=3mm]interior.south east)--++(-0.4,-0.1)--++(0.1,-0.2);
    \path[fill=yellow!50!black,draw=none] (interior.south west) rectangle node[white]{\Huge\bfseries !} ([xshift=4mm]interior.north west);
    },
  drop fuzzy shadow,#1}


\author{Akshay Dhan}
