% Some basic packages
\usepackage[utf8]{inputenc}
\usepackage[T1]{fontenc}
\usepackage{textcomp}
\usepackage{url}
\usepackage{graphicx}
\usepackage{float}
% \usepackage{enumitem}
\usepackage{enumerate}
\pdfminorversion=7


% Don't indent paragraphs, leave some space between them
\usepackage{parskip}

% Change the margins of the PDFs, PS: I like less margins.
\usepackage{geometry}
 \geometry{
 a4paper,
 total={175mm,263mm},
 left=15mm,
 top=15mm,
 }

% Hide page number when page is empty
\usepackage{emptypage}
\usepackage{subcaption}
\usepackage{multicol}
\usepackage{xcolor}

% Math stuff
\usepackage{amsmath, amsfonts, mathtools, amsthm, amssymb}
% Fancy script capitals
\usepackage{mathrsfs}
\usepackage{cancel}
% Bold math
\usepackage{bm}
% Some shortcuts
\DeclareMathAlphabet{\mathpzc}{OT1}{pzc}{m}{it}

\newcommand\x{\ensuremath{\mathpzc{x}}}
\newcommand\y{\ensuremath{\mathpzc{y}}}
\newcommand\z{\ensuremath{\mathpzc{z}}}
\newcommand\N{\ensuremath{\mathbb{N}}}
\newcommand\R{\ensuremath{\mathbb{R}}}
\newcommand\Z{\ensuremath{\mathbb{Z}}}
\renewcommand\O{\ensuremath{\emptyset}}
\newcommand\Q{\ensuremath{\mathbb{Q}}}
\newcommand\C{\ensuremath{\mathbb{C}}}
\newcommand\F{\ensuremath{\mathbb{F}}}
\newcommand{\Zn}{\ensuremath{\mathbb{Z}_{\geq 0}^n}}
\newcommand{\m}{\ensuremath{\mathfrak{m}}}
\renewcommand{\S}{\ensuremath{\mathcal{S}}}
\newcommand{\FF}{\ensuremath{\mathcal{F}}}
\renewcommand{\a}{\ensuremath{\mathfrak{a}}}
\renewcommand{\b}{\ensuremath{\mathfrak{b}}}
% \newcommand{\if}{\text{ if }}

%Make implies and impliedby shorter
\let\implies\Rightarrow
\let\impliedby\Leftarrow
\let\iff\Leftrightarrow
\let\epsilon\varepsilon
\let\phi\varphi
\DeclarePairedDelimiter{\ceil}{\lceil}{\rceil}
\DeclarePairedDelimiter{\floor}{\lfloor}{\rfloor}

% Add \contra symbol to denote contradiction
\usepackage{stmaryrd} % for \lightning
\newcommand\contra{\scalebox{1.5}{$\lightning$}}

% Command for short corrections
% Usage: 1+1=\correct{3}{2}

\definecolor{correct}{HTML}{009900}
\newcommand\correct[2]{\ensuremath{\:}{\color{red}{#1}}\ensuremath{\to }{\color{correct}{#2}}\ensuremath{\:}}
\newcommand\green[1]{{\color{correct}{#1}}}

% horizontal rule
\newcommand\hr{
    \noindent\rule[0.5ex]{\linewidth}{0.5pt}
}

% Environments
\makeatother
% For box around Definition, Theorem, \ldots
\usepackage{mdframed}
\mdfsetup{skipabove=1em,skipbelow=0em}
\theoremstyle{definition}
\newmdtheoremenv[nobreak=true]{lemma}{Lemma}
\newmdtheoremenv{conclusion}{Conclusion}
\newtheorem{defn}{Definition}
\newtheorem{algo}{Algorithm}
\newtheorem*{obs}{Observation}

\newmdtheoremenv[nobreak=true]{definition}{Definition}
\newtheorem*{eg}{Example}
\newtheorem*{notation}{Notation}
\newtheorem*{previouslyseen}{As previously seen}
\newtheorem*{rmk}{Remark}
\newtheorem*{note}{Note}
\newtheorem*{problem}{Problem}
\newtheorem*{observe}{Observe}
\newtheorem*{property}{Property}
\newtheorem*{intuition}{Intuition}
\newtheorem{prop}{Proposition}
\newmdtheoremenv[nobreak=true]{theorem}{Theorem}
\newtheorem{corollary}{Corollary}

\makeatletter
\def\thm@space@setup{%
  \thm@preskip=\parskip \thm@postskip=0pt
}

%a boxed title

% These are the fancy headers
\usepackage{fancyhdr}
\pagestyle{fancy}
\usepackage{xifthen}
\def\@lecture{}%
\newcommand{\lecture}[3]{
    \ifthenelse{\isempty{#3}}{%
        \def\@lecture{Lecture #1}%
    }{%
        \def\@lecture{Lecture #1: #3}%
    }%
    \subsection*{\@lecture}
    \pdfbookmark{\@lecture}{}
    \marginpar{\small\textsf{\mbox{#2}}}
}

\renewcommand{\headrulewidth}{0pt}
\fancyhead[RO,LE]{\@lecture} % Right odd,  Left even
\fancyhead[RE,LO]{}          % Right even, Left odd

\fancyfoot[RO,LE]{\thepage}  % Right odd,  Left even
\fancyfoot[RE,LO]{}          % Right even, Left odd
\fancyfoot[C]{\leftmark}     % Center

\makeatother
% Todonotes and inline notes in fancy boxes
\usepackage{todonotes}
\usepackage[most]{tcolorbox}
\usepackage{titling}

\newcommand{\makeboxtitle}{
	\noindent\fbox{ \parbox{\textwidth}{ \hfill \thedate \quad \begin{center}\Large \thetitle \end{center} \textsl{ \hfill \theauthor}}}\\~\\}

	\newcommand{\heading}[3]{
	\noindent\fbox{ \parbox{\textwidth}{  \hfill #1 \quad \begin{center}\Large Lecture #3 \end{center} \textsl{Lecturer: #2 \hfill Note taken: \theauthor}}}}


\newtcolorbox{noot}[1][]{enhanced,
  before skip=2mm,after skip=3mm,
  boxrule=0.4pt,left=5mm,right=2mm,top=1mm,bottom=1mm,
  colback=white!50,
  colframe=yellow!20!black,
  sharp corners,%rounded corners=southeast,arc is angular,arc=3mm,
  underlay={%
  %  \path[fill=black] ([yshift=3mm]interior.south east)--++(-0.4,-0.1)--++(0.1,-0.2);
%  \path[draw=black,shorten <=-0.05mm,shorten >=-0.05mm] ([yshift=3mm]interior.south east)--++(-0.4,-0.1)--++(0.1,-0.2);
    \path[fill=yellow!50!black,draw=none] (interior.south west) rectangle node[white]{\Huge\bfseries !} ([xshift=4mm]interior.north west);
    },
  drop fuzzy shadow,#1}


% Make boxes breakable
\tcbuselibrary{breakable}

\newenvironment{verbetering}{\begin{tcolorbox}[
    arc=0mm,
    colback=white,
    colframe=green!60!black,
    title=,
    fonttitle=\sffamily,
    breakable
]}{\end{tcolorbox}}

% Noot is note in Dutch. Same as 'verbetering' but color of box is different
%\newenvironment{noot}[1]{\begin{tcolorbox}[
%    arc=0mm,
%    colback=white,
%    colframe=white!60!black,
%    title=#1,
%    fonttitle=\sffamily,
%    breakable
%]}{\end{tcolorbox}}
%
\pdfsuppresswarningpagegroup=1

\usepackage[pdfusetitle]{hyperref}
\usepackage{import} %For adding diagrams made from inkscape
\newcommand{\incfig}[2]{%
    \def\svgwidth{#1}
    \import{./figures/}{#2.pdf_tex}
}

% My name
\author{Akshay Dhan}
